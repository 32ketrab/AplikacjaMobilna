\documentclass[a4paper]{article}
\usepackage{polski}
\usepackage[cp1250]{inputenc}
\usepackage{url}
\usepackage{graphicx}

\title{\bf{Kulinarna Aplikacja Mobilna}}
\author{{\em Poręba Bartłomiej, Zaczyk Bartłomiej, Szafrański Ireneusz}}
\date{}

\begin{document}
	
	\begin{titlepage}
		\maketitle
		\thispagestyle{empty}
		\bigskip
		\begin{center}
			Zespołowe przedsięwzięcie inżynierskie\\[2mm]
			
			Informatyka\\[2mm]
			
			Rok. akad. 2017/2018, sem. I\\[2mm]
			
			Prowadzący: dr hab. Marcin Mazur
		\end{center}
	\end{titlepage}
	
	\tableofcontents
	\thispagestyle{empty}
	
	\newpage
	
	\section{Opis projektu}
	
	\subsection{Członkowie zespołu}
	
	\begin{enumerate}
		\item Poręba Bartłomiej (kierownik projektu).
		\item Zaczyk Bartłomiej.
		\item Szafrański Ireneusz.
	\end{enumerate}
	
	\subsection{Cel projektu (produkt)}
	
	Celem projektu jest stworzenie aplikacji mobilnej, która na podstawie posiadanych przez użytkownika produktów znajdzie danie które użytkownik będzie mógł sporządzić, a ponadto będzie miał wgląd do wartości kalorycznych potrawy, przeglądać zdjęcia gotowej potrawy, będzie mógł przeczytać recenzje na temat danej potrawy oraz posortować listę wyszukanych potraw wg. swoich upodobań.
	
	\subsection{Potencjalny odbiorca produktu (klient)}
	
	Klientem może być każdy kto nie ma pomysłu na obiad i/lub nie umie gotować. Aplikacja w prosty i przystępny dla każdego sposób krok po kroku opisze jak sporządzić wybraną potrawę.
	
	\subsection{Metodyka}
	
	Projekt będzie realizowany przy użyciu (zaadaptowanej do istniejących warunków) metodyki {\em Scrum}. 
	
	\section{Wymagania użytkownika}
	<<Przedstawić listę wymagań użytkownika w postaci ,,historyjek'' (User stories). Każda historyjka powinna opisywać jedną cechę systemu. Struktura: As a [type of user], I want [to perform some task] so that I can [achieve some goal/benefit/value] (zob. np. \cite{us}).>>
	
	\subsection{User story 1}
	Jako użytkownik aplikacji chcę mieć dostęp do prostego i czytelnego interfejsu, żebym mógł szybko i sprawnie znaleźć to czego szukam. \newline
	*Aplikacja dostosowana do różnych rozmiarów urządzeń i systemów Android.\newline
	*Ograniczenie do minimum ilość przycisków.
	
	\subsection{User story 2}
	Jako użytkownik aplikacji chcę wiedzieć co mogę sporządzić z produktów, które posiadam, żebym nie tracił czasu na poszukiwania przepisów. \newline
	*Baza danych regularnie aktualizowana, aby zapewnić szeroki wybór przepisów.
	
	\subsection{User story 3}
	Jako użytkownik aplikacji chcę wiedzieć ile kalorii ma dana potrawa, żebym mógł kontrolować to co jem. \newline
	*Opcja liczenia kalorii dla pojedyńczego składniku i całej potrawy.
	
	\subsection{User story 4}
	Jako użytkownik aplikacji chcę sprawdzić jakich produktów brakuje mi do potrawy, żebym mógł delektować się jej smakiem. \newline
	*Opcja sortowania potraw wg ilości brakujących składników.
	
	\subsection{User story 5}
	Jako użytkownik aplikacji chcę żeby aplikacja na telefon nie zajmowała dużo miejsca, żebym mógł mieć na nim miejsce na zdjęcia i filmy. \newline
	*Baza danych online, przez co mniejszy rozmiar aplikacji .
	
	\subsection{User story 6}
	Jako użytkownik aplikacji chcę wiedzieć ile czasu potrzebuje na sporządzenie danej potrawy, żebym mógł odpowiednio zagospodarować swój wolny czas. \newline
	*Przy każdej potrawie jest szacowany czas jej przygotowania.
	
	\subsection{User story 7}
	Jako użytkownik aplikacji chcę mieć dostęp do ulubionych przepisów kiedy jestem offline, żebym mógł korzystać z niej gdziekolwiek jestem. \newline
	*Możliwość zapisywania ulubionych przepisów.
	
	\subsection{User story 8}
	Jako użytkownik aplikacji chcę wiedzieć jak inni użytkownicy oceniają daną potrawę, żebym mógł przygotować danie które oni polecają. \newline
	*Możliwość oceniania potraw i dzielenia się opinią z innymi użytkownikami aplikacji.
	
	\subsection{User story 9}
	Jako użytkownik aplikacji chcę dzielić się moimi przepisami, żebym mógł poznać opinie na jej temat. \newline
	*Możliwość dodawania przepisów do ogólnej bazy danych (po zatwierdzeniu).
	
	
	\section{Harmonogram}
	
	\subsection{Rejestr zadań (Product Backlog)}
	
	\begin{itemize}
		\item Data rozpoczęcia: 03.10.2017.
		\item  Data zakończenia: 07.11.2017.
	\end{itemize}
	
	\subsection{Sprint 1}
	
	\begin{itemize}
		\item Data rozpoczęcia: 07.11.2017.
		\item Data zakończenia: 21.11.2017.
		\item Scrum Master: Poręba Bartłomiej.
		\item Product Owner: Zaczyk Bartłomiej.
		\item Development Team: Poręba Bartłomiej, Zaczyk Bartłomiej, Szafrański Ireneusz.
	\end{itemize}
	
	\subsection{Sprint 2}
	
	\begin{itemize}
		\item Data rozpoczęcia: 21.11.2017.
		\item  Data zakończenia: 5.12.2017.
		\item Scrum Master: Zaczyk Bartłomiej.
		\item Product Owner: Poręba Bartłomiej.
		\item Development Team: Poręba Bartłomiej, Zaczyk Bartłomiej, Szafrański Ireneusz.
	\end{itemize}
	
	\subsection{Sprint 3}
	
	\begin{itemize}
		\item Data rozpoczęcia: 05.12.2017.
		\item  Data zakończenia: 19.12.2017.
		\item Scrum Master: Poręba Bartłomiej.
		\item Product Owner: Szafrański Ireneusz.
		\item Development Team: Poręba Bartłomiej, Zaczyk Bartłomiej, Szafrański Ireneusz.
	\end{itemize}
	
	\subsection{Sprint 4}
	
	\begin{itemize}
		\item Data rozpoczęcia: 19.12.2017.
		\item  Data zakończenia: 02.01.2018.
		\item Scrum Master: Zaczyk Bartłomiej.
		\item Product Owner: Szafrański Ireneusz.
		\item Development Team: Poręba Bartłomiej, Zaczyk Bartłomiej, Szafrański Ireneusz.
	\end{itemize}
	
	\subsection{Sprint 5}
	
	\begin{itemize}
		\item Data rozpoczęcia: 02.01.2018.
		\item  Data zakończenia: 23.01.2018.
		\item Scrum Master: Szafrański Ireneusz.
		\item Product Owner: Poręba Bartłomiej.
		\item Development Team: Poręba Bartłomiej, Zaczyk Bartłomiej, Szafrański Ireneusz.
	\end{itemize}
	
	
	\section{Product Backlog}
	
	\subsection{Backlog Item 1}
	\paragraph{Tytuł zadania.} Wstępne zaprogramowanie aplikacji.
	\paragraph{Opis zadania.} Konfiguracja środowiska MIT App Inventor i zapoznaniem się z jego funkcjami oraz zdefiniowanie wszystkich przycisków w taki sposób, aby każdy odsyłał do odpowiednie dla niego miejsce zgodnie z mapą aplikacji.
	\paragraph{Priorytet.} 1.
	\paragraph{Definition of Done.} Stworzenie mapy (szkielet) aplikacji oraz przycisków, które będą odsyłać w odpowiednie miejsce na mapie
	
	\subsection{Backlog Item 2}
	\paragraph{Tytuł zadania.} Projektowanie prostego i czytelnego interfejsu graficznego. (\textit{user story 1})
	\paragraph{Opis zadania.} Graficzne zaprojektowanie przejrzystego interfejsu umożliwiającego proste i intuicyjne korzystanie z aplikacji.
	\paragraph{Priorytet.} 2.
	\paragraph{Definition of Done.} Przejrzysty i czytelny interfejs graficzny wdrożony do aplikacji za pomocą programu MIT App Inventor.
	
	\subsection{Backlog Item 3}
	\paragraph{Tytuł zadania.} Stworzenie bazy danych składającą się z listy przepisów i składników (\textit{user story 2 , 3 , 6})
	\paragraph{Opis zadania.} Baza danych postawiona na serwerze mysql. Baza zawiera spis potraw oraz składników, jakie potrzeba do jej sporządzenia, dodanie do każdej z potraw wartości kalorycznej oraz szacowanego czasu przyżądzenia danej potrawy
	\paragraph{Priorytet.} 2.
	\paragraph{Definition of Done.} Użytkownik jest w stanie na podstawie zawartych informacji o produktach w opcji "Szukaj" wyszukać potrawę którą może z nich sporządzić oraz po wyszukaniu jej będzie miał wgląd do listy potraw, wartości kalorycznej każdej z nich oraz szacowanego czasu jej przygotowania.
	
	\subsection{Backlog Item 4}
	\paragraph{Tytuł zadania.} Zaprogramowanie opcji "Szukaj" oraz wdrożenie jej do aplikacji.
	\paragraph{Opis zadania.} Zaprogramowanie opcji "Szukaj" w taki sposób, aby użytkownik mógł wybrać składniki które posiada, na podstawie których w kolejnym Sprincie będą wyszukiwane potrawy.
	\paragraph{Priorytet.} 3.
	\paragraph{Definition of Done.} Użytkownik ma dostęp do listy składników z których może wybrać te, które posiada lub z których chce stworzyć potrawe.
	
	\subsection{Backlog Item 5}
	\paragraph{Tytuł zadania.} Wdrożenie do aplikacji opcji sortowania. (\textit{user story 4})
	\paragraph{Opis zadania.} Zaprogramowanie wyników wyszukiwania w taki sposób, aby użytkownik mógł wybrać opcję sortowania wyników wyszukiwania wg wartości kalorycznej lub ilości brakujących składników
	\paragraph{Priorytet.} 3.
	\paragraph{Definition of Done.} Użytkownik będzie mógł sortować wyniki wyszukiwania wg brakujących składników (od tych przepisów w których jest komplet po te gdzie brakuje tych składników najwięcej)
	
	\subsection{Backlog Item 6}
	\paragraph{Tytuł zadania.} Wstawienie Bazy danych na domene (serwer - w przypadku prezentacji). (\textit{user story 5})
	\paragraph{Opis zadania.} Istniejącą baze danych zaimplementować na domenie (na serwerze w przypadku prezentacji) oraz przestawić źródło odczytu danych na tę domene/serwer
	\paragraph{Priorytet.} 4.
	\paragraph{Definition of Done.} Użytkownik może wyszukiwać przepisy z bazy danych umieszczonej na domenie/serwerze.
	
	\subsection{Backlog Item 7}
	\paragraph{Tytuł zadania.} Wdrożenie opcji zapisywania przepisów offline. (\textit{user story 7})
	\paragraph{Opis zadania.} Do każdej potrawy zostanie dodany przycisk umożliwiający zapisanie przepisu na daną potrawe do pliku, z której następnie użytkownik będzie mógł kożystać wtedy kiedy nie jest podłączony do internetu
	\paragraph{Priorytet.} 4.
	\paragraph{Definition of Done.} Użytkownik ma dostęp do wybranych potraw kiedy nie jest połączony z internetem poprzez plik o specjalnym rozszerzeniu.
	
	\subsection{Backlog Item 8}
	\paragraph{Tytuł zadania.} Wprowadzenie systemu oceny i recenzji. (\textit{user story 8})
	\paragraph{Opis zadania.} Zaprojektowanie systemu recenzji i oceny oraz wdrożenie jej do aplikacji.
	\paragraph{Priorytet.} 5.
	\paragraph{Definition of Done.} Każdy użytkownik aplikacji będzie mógł recenzować daną potrawę oraz ją oceniać.
	
	\subsection{Backlog Item 9}
	\paragraph{Tytuł zadania.} Wprowadzenie do aplikacji opcji Supportu. (\textit{user story 9})
	\paragraph{Opis zadania.} Dodanie do aplikacji zakładki "support" przez którą użytkownicy aplikacji będą mogli zgłaszać jakiekolwiek problemy z aplikacją oraz wysyłać swoje własne przepisy, które po zweryfikowaniu zostaną dodane do bazy przepisów
	\paragraph{Priorytet.} 5.
	\paragraph{Definition of Done.} Użytkownik jest w stanie wysyłać wiadomości do administratora aplikacji na temat problemów z aplikacją lub z prośbą o dodanie przepisu do aplikacji
	
	
	\section{Sprint 1}
	\subsection{Cel} Wstępny projekt aplikacji oraz stworzenie siatki umożliwiającej dostęp do różnych ekranów aplikacji.
	\subsection{Sprint Planning/Backlog}
	
	\paragraph{Tytuł zadania.} Wstępne zaprogramowanie aplikacji.
	\begin{itemize}
		\item Estymata: M.
	\end{itemize}
	
	\paragraph{Tytuł zadania.} Projektowanie prostego i czytelnego interfejsu graficznego.
	\begin{itemize}
		\item Estymata: L.
	\end{itemize}
	
	\subsection{Realizacja}
	
	\paragraph{Tytuł zadania.} Wstępne zaprogramowanie aplikacji.
	\subparagraph{Wykonawca.} Poręba Bartłomiej, Zaczyk Bartłomiej, Szafrański Ireneusz.
	\subparagraph{Realizacja.} Na początku sprintu wybraliśmy program Android Studio. Zapewnia on wiele możliwości, jednak aby sprawnie się nim obsługiwać potrzebna jest wszechstronna wiedza, której nie mieliśmy. Po Konsultacji postanowiliśmy zmienić program w którym będziemy tworzyć aplikację na MIT APP Inventor. Okazał się on bardzo intuicyjny i po kilku filmach instruktarzowych potrafiliśmy z nigo korzystać. W grupie ustaliliśmy ogólną wizję jak ma wyglądać nasza aplikacja i po obradach stworzyliśmy szkielet, który sprostał naszym oczekiwaniom.
	
	\paragraph{Tytuł zadania.} Projektowanie prostego i czytelnego interfejsu graficznego.
	\subparagraph{Wykonawca.} Poręba Bartłomiej, Zaczyk Bartłomiej, Szafrański Ireneusz.
	\subparagraph{Realizacja.}
	Po stworzeniu szkieletu aplikacji rozpoczęliśmy etap tworzenia oprawy graficznej; Bartek Zaczyk zajął się Ekranem głównym, Ireneusz Szafrański był odpowiedzialny za ekran Przeglądaj a Bartek Poręba za ekran Szukaj oraz złożył to wszystko w całość. Pierwszym problemem jaki napotkaliśmy było zmienienie funkcji .ShowMessageDialog na ekranie Szukaj w taki sposób aby wyświetlony komunikat posiadał pole do wpisania ilości danego składnika. Problem ten nie został rozwiązany, dlatego zostanie on rozwiązany w kolejnym sprincie. Przy ekranie Przeglądaj musieliśmy przedyskutować problem wyświetlania przepisów w taki sposób, aby było to czytelne dla użytkownika, dlatego oprócz standardowego wyświetlania przepisów postanowiliśmy dodać opcje "Wyświetlania na stronie", która będzie wyświetlała wskazaną przez użytkownika ilość produktów, która po naciśnięciu przycisku "Dalej" odświerzy stronę z nowymi przepisami.
	
	\subsection{Sprint Review/Demo}
	Sprint został wykonany oraz cel sprintu został osiągnięty.
	\begin{figure}
		\centering
		\includegraphics[width=15cm]{1.jpg}
		\caption{Ekran Główny}
		\label{glowny}
	\end{figure}
	
	\begin{figure}
		\centering
		\includegraphics[width=15cm]{2.jpg}
		\caption{Przeglądaj}
		\label{przegladaj}
	\end{figure}

	\begin{figure}
		\centering
		\includegraphics[width=15cm]{3.jpg}
		\caption{Szukaj}
		\label{Szukaj}
	\end{figure}
	
	\section{Sprint 2}
	
	\subsection{Cel} Stworzenie bazy danych z przepisami, dołączenie jej do aplikacji oraz zaprogramowanie opcji "Szukaj".
	
	\subsection{Sprint Planning/Backlog}
	
	\paragraph{Tytuł zadania.} Stworzenie bazy danych składającą się z listy przepisów i składników
	\begin{itemize}
		\item Estymata: L.
	\end{itemize}
	
	\paragraph{Tytuł zadania.} Zaprogramowanie opcji "Szukaj" oraz wdrożenie jej do aplikacji.
	\begin{itemize}
		\item Estymata: L.
	\end{itemize}
	
	\paragraph{Tytuł zadania.} Stworzenie bazy danych i dołączenie jej do aplikacji.
	\subparagraph{Wykonawca.} Poręba Bartłomiej, Zaczyk Bartłomiej, Szafrański Ireneusz.
	\subparagraph{Realizacja.} 
	Po wstępnych oględzinach przepisów na różne potrawy (głównie obiadowych) postanowiliśmy wstępnie stworzyć kilka przepisów na ciasta, gdyż przepisy na nie są w pewnym sensie powtarzalne, przez co baza składników nie będzie zbyt duża, w wyniku czego łatwiej będzie zaprogramować całą aplikacje. W przyszłych sprintach mamy zaplanowane dodawanie przepisów przez użytkowników, więc baza będzie się rozwijała na zasadzie "Crowdsourcing'owej". Baza składa się z baz wszystkich składników, bazy składników potrzebnych do sporządzenia potrawy oraz bazy samych przepisów (instrukcji jak ją zrobić). Baza inicjuje się po kliknięciu w przycisk "Szukaj".
	

	\paragraph{Tytuł zadania.} Zaprogramowanie opcji "Szukaj".
	\subparagraph{Wykonawca.} Poręba Bartłomiej, Zaczyk Bartłomiej, Szafrański Ireneusz.
	\subparagraph{Realizacja.}
	Na rozpoczęciu sprintu zaplanowaliśmy jak będzie działąła opcja szukaj, oraz staraliśmy się znależć odpowiedni serwis oferujący darmowe stworzenie bazy danych w internecie. Wstępnie ustaliliśmy, że podzielimy bazę danych na dwa - Składniki oraz Przepisy. Wybrane przez użytkownika składniki będą podstawą do wyszukiwania całej bazy przepisów, która w zmiennej "ilosc-brakujacych-skladnikow" będzie przy każdej potrawie zapisywała ile składników brakuje. Jest to niezbędne, aby przeszukac całą baze przepisów (nawet jeśli znajduje się w niej tysiąc przepisów), ponieważ może się tak okazać, że jeśli najpierw przeszuka pierwsze 100 przepisów i posortuje je wg ilości brakujacych składników to może się okazać, że w kolejnej "setce" przepisów może znależć się taki, któremu brakuję mniej składników.
	
	
	
	\subsection{Sprint Review/Demo}
W sprincie niestety nie udało się nam stworzyć kompletnej bazy danych ze względu na brak wystarczającej ilości czasu, więc skupiliśmy się na tym jak aplikacja będzie odczytywała podane przez użytkownika składniki oraz jak będzie porównywać je z listą składników każdej potrawy.
	
	
	\section{Sprint 3}
	
	\subsection{Cel} Wdrożenie do aplikacji opcjii sortowania oraz wstawienie bazy danych na domene.
	
	\subsection{Sprint Planning/Backlog}
	
	\paragraph{Tytuł zadania.} Wdrożenie do aplikacji opcji sortowania.
	\begin{itemize}
		\item Estymata: \textbf{XL}.
	\end{itemize}
	\subsection{Realizacja}
	Na rozpoczęciu sprintu ustaliliśmy zespołowo, że należy nadrobić zaległości z poprzedniego sprintu (w końcu, aby wdrożyć opcje sortowania trzeba najpierw mieć bazę conajmniej kilku przepisów oraz zoprymalizowane opcje liczenia brakujących składników). Dlatego przez ostatni tydzień skupiliśmy się na rozwiązaniu problemu jakim jest wprowadzanie przez użytkownika ilości posiadanych składników. W tym celu stworzyliśmy wstępną bazę przepisów składającą się z trzech potraw i ustaliliśmy, że kolejne przepisy będziemy dodawać z czasem (dodawanie jest intuicyjne). Poza stworzeniem bazy kilku przepisów ustabilizowaliśmy system zliczający ilość brakujących składników, araz funkcję wypisującą w szczegółach przepisu jakich i ile składników brakuje. Stworzyliśmy jeszcze ogólny podgląd przepisu, w którym jest wyżej wymieniona funkcja, którą w gronie zespołu nazwaliśmy "lista zakupów" (brakujące składniki). Przez następny tydzień skupimy się na realizacji obecnego sprintu.
	
	
	\paragraph{Tytuł zadania.} Wstawienie Bazy danych na domene (serwer - w przypadku prezentacji).
	\begin{itemize}
		\item Estymata: L.
	\end{itemize}
	
	
	\subsection{Realizacja}
	
	
	
	\subsection{Sprint Review/Demo}
	<<Sprawozdanie z przeglądu Sprint'u -- czy założony cel (przyrost) został osiągnięty oraz czy wszystkie zaplanowane Backlog Item'y zostały zrealizowane? Demostracja przyrostu produktu>>.
	
	
	\section{Sprint 4}
	
	\subsection{Cel} Wdrożenie do aplikacji opcji zapisywania przepisów offline oraz systemu oceny i recenzji.
	
	\subsection{Sprint Planning/Backlog}
	
	\paragraph{Tytuł zadania.} Wdrożenie opcji zapisywania przepisów offline.
	\begin{itemize}
		\item Estymata: L.
	\end{itemize}
	
	\paragraph{Tytuł zadania.} Wprowadzenie systemu oceny i recenzji.
	\begin{itemize}
		\item Estymata: M.
	\end{itemize}
	
	\paragraph{<<Tutaj dodawać kolejne zadania>>}
	
	\subsection{Realizacja}
	
	\paragraph{Tytuł zadania.} Zaprogramowanie opcji "Szukaj".
	\subparagraph{Wykonawca.} Poręba Bartłomiej, Zaczyk Bartłomiej, Szafrański Ireneusz.
	\subparagraph{Realizacja.} <<Sprawozdanie z realizacji zadania (w tym ocena zgodności z estymatą). Kod programu (środowisko \texttt{verbatim}): \begin{verbatim}
	for (i=1; i<10; i++)
	...
	\end{verbatim}>>.
	
	\paragraph{Tytuł zadania.} Wdrożenie systemu liczenia kalorii.
	\subparagraph{Wykonawca.} Poręba Bartłomiej, Zaczyk Bartłomiej, Szafrański Ireneusz.
	\subparagraph{Realizacja.} <<Sprawozdanie z realizacji zadania (w tym ocena zgodności z estymatą). Kod programu (środowisko \texttt{verbatim}): \begin{verbatim}
	for (i=1; i<10; i++)
	...
	\end{verbatim}>>.
	
	
	
	\subsection{Sprint Review/Demo}
	<<Sprawozdanie z przeglądu Sprint'u -- czy założony cel (przyrost) został osiągnięty oraz czy wszystkie zaplanowane Backlog Item'y zostały zrealizowane? Demostracja przyrostu produktu>>.
	
	
	\section{Sprint 5}
	
	\subsection{Cel} Wprowadzenie możliwości recenzji, oceny oraz zapisywania przepisów w pamięci urządzenia.
	
	\subsection{Sprint Planning/Backlog}
	
	\paragraph{Tytuł zadania.} Wprowadzenie do aplikacji opcji Supportu.
	\begin{itemize}
		\item Estymata: M.
	\end{itemize}
	
	
	\subsection{Realizacja}
	
	\paragraph{Tytuł zadania.} Wprowadzenie do aplikacji opcji Supportu.
	\subparagraph{Wykonawca.} Poręba Bartłomiej, Zaczyk Bartłomiej, Szafrański Ireneusz.
	\subparagraph{Realizacja.} <<Sprawozdanie z realizacji zadania (w tym ocena zgodności z estymatą). Kod programu (środowisko \texttt{verbatim}): \begin{verbatim}
	for (i=1; i<10; i++)
	...
	\end{verbatim}>>.
	
	\paragraph{Tytuł zadania.} Wdrożenie opcji zapisywania przepisów offline.
	\subparagraph{Wykonawca.} Poręba Bartłomiej, Zaczyk Bartłomiej, Szafrański Ireneusz.
	\subparagraph{Realizacja.} <<Sprawozdanie z realizacji zadania (w tym ocena zgodności z estymatą). Kod programu (środowisko \texttt{verbatim}): \begin{verbatim}
	for (i=1; i<10; i++)
	...
	\end{verbatim}>>.
	
	
	\subsection{Sprint Review/Demo}
	<<Sprawozdanie z przeglądu Sprint'u -- czy założony cel (przyrost) został osiągnięty oraz czy wszystkie zaplanowane Backlog Item'y zostały zrealizowane? Demostracja przyrostu produktu>>.
	
	
	
	\begin{thebibliography}{9}
		
		\bibitem{Cov} S. R. Covey, {\em 7 nawyków skutecznego działania}, Rebis, Poznań, 2007.
		
		\bibitem{Oet} Tobias Oetiker i wsp., Nie za krótkie wprowadzenie do systemu \LaTeX  \ $2_\varepsilon$, \url{ftp://ftp.gust.org.pl/TeX/info/lshort/polish/lshort2e.pdf}
		
		\bibitem{SchSut} K. Schwaber, J. Sutherland, {\em Scrum Guide}, \url{http://www.scrumguides.org/}, 2016.
		
		\bibitem{apr} \url{https://agilepainrelief.com/notesfromatooluser/tag/scrum-by-example}
		
		\bibitem{us} \url{https://www.tutorialspoint.com/scrum/scrum_user_stories.htm}
		
	\end{thebibliography}
	
\end{document}

% ----------------------------------------------------------------
