\documentclass[a4paper]{article}
\usepackage{polski}
\usepackage[cp1250]{inputenc}
\usepackage{url}

\title{\bf{Kulinarna Aplikacja Mobilna}}
\author{{\em Poręba Bartłomiej, Zaczyk Bartłomiej, Szafrański Ireneusz}}
\date{}

\begin{document}

\begin{titlepage}
\maketitle
\thispagestyle{empty}
\bigskip
\begin{center}
Zespołowe przedsięwzięcie inżynierskie\\[2mm]

Informatyka\\[2mm]

Rok. akad. 2017/2018, sem. I\\[2mm]

Prowadzący: dr hab. Marcin Mazur
\end{center}
\end{titlepage}

\tableofcontents
\thispagestyle{empty}

\newpage

\section{Opis projektu}

\subsection{Członkowie zespołu}

\begin{enumerate}
\item Poręba Bartłomiej (kierownik projektu).
\item Zaczyk Bartłomiej.
\item Szafrański Ireneusz.
\end{enumerate}

\subsection{Cel projektu (produkt)}

Celem projektu jest stworzenie aplikacji mobilnej, która na podstawie posiadanych przez użytkownika produktów znajdzie danie które użytkownik będzie mógł sporządzić, a ponadto będzie miał wgląd do wartości kalorycznych potrawy, przeglądać zdjęcia gotowej potrawy, będzie mógł przeczytać recenzje na temat danej potrawy oraz posortować listę wyszukanych potraw wg. swoich upodobań.

\subsection{Potencjalny odbiorca produktu (klient)}

Klientem może być każdy kto nie ma pomysłu na obiad i/lub nie umie gotować. Aplikacja w prosty i przystępny dla każdego sposób krok po kroku opisze jak sporządzić wybraną potrawę.

\subsection{Metodyka}

Projekt będzie realizowany przy użyciu (zaadaptowanej do istniejących warunków) metodyki {\em Scrum}. 

\section{Wymagania użytkownika}
<<Przedstawić listę wymagań użytkownika w postaci ,,historyjek'' (User stories). Każda historyjka powinna opisywać jedną cechę systemu. Struktura: As a [type of user], I want [to perform some task] so that I can [achieve some goal/benefit/value] (zob. np. \cite{us}).>>

\subsection{User story 1}
Jako użytkownik aplikacji chcę mieć dostęp do prostego i czytelnego interfejsu, żebym mógł szybko i sprawnie znaleźć to czego szukam. \newline
*Aplikacja dostosowana do różnych rozmiarów urządzeń i systemów Android.\newline
*Ograniczenie do minimum ilość przycisków.

\subsection{User story 2}
Jako użytkownik aplikacji chcę wiedzieć co mogę sporządzić z produktów, które posiadam, żebym nie tracił czasu na poszukiwania przepisów. \newline
*Baza danych regularnie aktualizowana, aby zapewnić szeroki wybór przepisów.

\subsection{User story 3}
Jako użytkownik aplikacji chcę wiedzieć ile kalorii ma dana potrawa, żebym mógł kontrolować to co jem. \newline
*Opcja liczenia kalorii dla pojedyńczego składniku i całej potrawy.

\subsection{User story 4}
Jako użytkownik aplikacji chcę sprawdzić jakich produktów brakuje mi do potrawy, żebym mógł delektować się jej smakiem. \newline
*Opcja sortowania potraw wg ilości brakujących składników.

\subsection{User story 5}
Jako użytkownik aplikacji chcę wiedzieć jak inni użytkownicy oceniają daną potrawę, żebym mógł przygotować danie które oni polecają. \newline
*Możliwość oceniania potraw i dzielenia się opinią z innymi użytkownikami aplikacji.

\subsection{User story 6}
Jako użytkownik aplikacji chcę żeby aplikacja na telefon nie zajmowała dużo miejsca, żebym mógł mieć na nim miejsce na zdjęcia i filmy. \newline
*Baza danych online, przez co mniejszy rozmiar aplikacji .

\subsection{User story 7}
Jako użytkownik aplikacji chcę wiedzieć ile czasu potrzebuje na sporządzenie danej potrawy, żebym mógł odpowiednio zagospodarować swój wolny czas. \newline
*Przy każdej potrawie jest szacowany czas jej przygotowania.

\subsection{User story 8}
Jako użytkownik aplikacji chcę mieć dostęp do ulubionych przepisów kiedy jestem offline, żebym mógł korzystać z niej gdziekolwiek jestem. \newline
*Możliwość zapisywania ulubionych przepisów.

\subsection{User story 9}
Jako użytkownik aplikacji chcę dzielić się moimi przepisami, żebym mógł poznać opinie na jej temat. \newline
*Możliwość dodawania przepisów do ogólnej bazy danych (po zatwierdzeniu).


\section{Harmonogram}

\subsection{Rejestr zadań (Product Backlog)}

\begin{itemize}
\item Data rozpoczęcia: 31.10.2017.
\item  Data zakończenia: 23.01.2018.
\end{itemize}

\subsection{Sprint 1}

\begin{itemize}
\item Data rozpoczęcia: 07.11.2017.
\item Data zakończenia: 14.11.2017.
\item Scrum Master: Poręba Bartłomiej.
\item Product Owner: Zaczyk Bartłomiej.
\item Development Team: Poręba Bartłomiej, Zaczyk Bartłomiej, Szafrański Ireneusz.
\end{itemize}

\subsection{Sprint 2}

\begin{itemize}
\item Data rozpoczęcia: 14.11.2017.
\item  Data zakończenia: 28.11.2017.
\item Scrum Master: Zaczyk Bartłomiej.
\item Product Owner: Poręba Bartłomiej.
\item Development Team: Poręba Bartłomiej, Zaczyk Bartłomiej, Szafrański Ireneusz.
\end{itemize}

\subsection{Sprint 3}

\begin{itemize}
	\item Data rozpoczęcia: 28.11.2017.
	\item  Data zakończenia: 26.12.2017.
	\item Scrum Master: Poręba Bartłomiej.
	\item Product Owner: Szafrański Ireneusz.
	\item Development Team: Poręba Bartłomiej, Zaczyk Bartłomiej, Szafrański Ireneusz.
\end{itemize}

\subsection{Sprint 4}

\begin{itemize}
	\item Data rozpoczęcia: 26.12.2017.
	\item  Data zakończenia: 09.01.2018.
	\item Scrum Master: Zaczyk Bartłomiej.
	\item Product Owner: Szafrański Ireneusz.
	\item Development Team: Poręba Bartłomiej, Zaczyk Bartłomiej, Szafrański Ireneusz.
\end{itemize}

\subsection{Sprint 5}

\begin{itemize}
	\item Data rozpoczęcia: 09.01.2018.
	\item  Data zakończenia: 23.01.2018.
	\item Scrum Master: Szafrański Ireneusz.
	\item Product Owner: Poręba Bartłomiej.
	\item Development Team: Poręba Bartłomiej, Zaczyk Bartłomiej, Szafrański Ireneusz.
\end{itemize}


\section{Product Backlog}

\subsection{Backlog Item 1}
\paragraph{Tytuł zadania.} Wstępne zaprogramowanie aplikacji.
\paragraph{Opis zadania.} Instalacja środowiska Android Studio wraz z konfiguracją i zapoznaniem się z jego funkcjami oraz zdefiniowanie wszystkich przycisków w taki sposób, aby każdy odsyłał do odpowiedniego dla niego ekranu.
\paragraph{Priorytet.} 1.
\paragraph{Definition of Done.} Każdy przycisk odsyła w odpowiednie dla niego miejsce.

\subsection{Backlog Item 2}
\paragraph{Tytuł zadania.} Projektowanie prostego i czytelnego interfejsu graficznego.
\paragraph{Opis zadania.} Wizualne projektowanie przejrzystego interfejsu.
\paragraph{Priorytet.} 2.
\paragraph{Definition of Done.} Kompletny i sprawdzony interfejs stworzony w aplikacji Android Studio.

\subsection{Backlog Item 3}
\paragraph{Tytuł zadania.} Stworzenie bazy danych i dołączenie jej do aplikacji.
\paragraph{Opis zadania.} Baza danych postawiona na serwerze mysql. Baza uzupełniona danymi. Zaimplementowanie bazy do aplikacji.
\paragraph{Priorytet.} 2.
\paragraph{Definition of Done.} Aplikacja widzi i w prawidłowy sposób odczytuje dane zawarte w bazie banych.

\subsection{Backlog Item 4}
\paragraph{Tytuł zadania.} Zaprogramowanie opcji "Szukaj".
\paragraph{Opis zadania.} Wprowadzienie do aplikacji odpowiednich zmiennych do wyszukiwania.
\paragraph{Priorytet.} 3.
\paragraph{Definition of Done.} Po wprowadzeniu produktów do wyszukiwarki aplikacja podaje listę przepisów, które można z nich sporządzić oraz sortuje je wg ilości brakujących składników (najpierw pełne przepisy, następnie z brakującym jednym składnikiem itd...).

\subsection{Backlog Item 5}
\paragraph{Tytuł zadania.} Wdrożenie systemu liczenia kalorii.
\paragraph{Opis zadania.} Dodanie wartości kalorycznej do każdego składnika oraz zaprogramowanie aplikacji w taki sposów, aby liczyła końcową wartość kaloryczną dla całej potrawy.
\paragraph{Priorytet.} 4.
\paragraph{Definition of Done.} Każdy przepis ma wyświetlone informacje o wartościach kalorycznych potrawy.

\subsection{Backlog Item 6}
\paragraph{Tytuł zadania.} Wprowadzenie systemu oceny i recenzji.
\paragraph{Opis zadania.} Zaprojektowanie systemu recenzji i oceny oraz wdrożenie jej do aplikacji.
\paragraph{Priorytet.} 4.
\paragraph{Definition of Done.} Każdy użytkownik aplikacji będzie mógł recenzować daną potrawę oraz ją oceniać.

\subsection{Backlog Item 7}
\paragraph{Tytuł zadania.} Wdrożenie opcji zapisywania przepisów offline.
\paragraph{Opis zadania.} Dodanie opcji zapisu wybranego przez użytkownika przepisu na pamięć swojego telefonu.
\paragraph{Priorytet.} 5.
\paragraph{Definition of Done.} Użytkownik ma dostęp do ulubionych potraw kiedy nie jest połączony z internetem.
.
\subsection*{<<Tutaj dodawać kolejne zadania>>}

\section{Sprint 1}
\subsection{Cel} Wstępny projekt aplikacji oraz stworzenie siatki umożliwiającej dostęp do różnych ekranów aplikacji.
\subsection{Sprint Planning/Backlog}

\paragraph{Tytuł zadania.} Wstępne zaprogramowanie aplikacji.
\begin{itemize}
	\item Estymata: L.
\end{itemize}


\subsection{Realizacja}

\paragraph{Tytuł zadania.} Wstępne zaprogramowanie aplikacji.
\subparagraph{Wykonawca.} Poręba Bartłomiej, Zaczyk Bartłomiej, Szafrański Ireneusz.
\subparagraph{Realizacja.} <<Sprawozdanie z realizacji zadania (w tym ocena zgodności z estymatą). Kod programu (środowisko \texttt{verbatim}): \begin{verbatim}
for (i=1; i<10; i++)
...
\end{verbatim}>>.


\subsection{Sprint Review/Demo}
<<Sprawozdanie z przeglądu Sprint'u -- czy założony cel (przyrost) został osiągnięty oraz czy wszystkie zaplanowane Backlog Item'y zostały zrealizowane? Demostracja przyrostu produktu>>.


\section{Sprint 2}

\subsection{Cel} Prosty w obsłudze i w pełni funkcjonalny interfejs graficzny wraz ze wstępnym zaprogramowaniem aplikacji.

\subsection{Sprint Planning/Backlog}

\paragraph{Tytuł zadania.} Projektowanie prostego i czytelnego interfejsu graficznego.
\begin{itemize}
\item Estymata: L.
\end{itemize}

\subsection{Realizacja}

\paragraph{Tytuł zadania.} Projektowanie prostego i czytelnego interfejsu graficznego.
\subparagraph{Wykonawca.} Poręba Bartłomiej, Zaczyk Bartłomiej, Szafrański Ireneusz.
\subparagraph{Realizacja.} <<Sprawozdanie z realizacji zadania (w tym ocena zgodności z estymatą). Kod programu (środowisko \texttt{verbatim}): \begin{verbatim}
for (i=1; i<10; i++)
...
\end{verbatim}>>.


\subsection{Sprint Review/Demo}
<<Sprawozdanie z przeglądu Sprint'u -- czy założony cel (przyrost) został osiągnięty oraz czy wszystkie zaplanowane Backlog Item'y zostały zrealizowane? Demostracja przyrostu produktu>>.


\section{Sprint 3}

\subsection{Cel} Stworzenie bazy danych z przepisami oraz zaimplementowanie jej do aplikacji.

\subsection{Sprint Planning/Backlog}

\paragraph{Tytuł zadania.} Stworzenie bazy danych i dołączenie jej do aplikacji.
\begin{itemize}
	\item Estymata: \textbf{XL}.
\end{itemize}


\subsection{Realizacja}

\paragraph{Tytuł zadania.}  Stworzenie bazy danych i dołączenie jej do aplikacji.
\subparagraph{Wykonawca.} Poręba Bartłomiej, Zaczyk Bartłomiej, Szafrański Ireneusz.
\subparagraph{Realizacja.} <<Sprawozdanie z realizacji zadania (w tym ocena zgodności z estymatą). Kod programu (środowisko \texttt{verbatim}): \begin{verbatim}
for (i=1; i<10; i++)
...
\end{verbatim}>>.



\subsection{Sprint Review/Demo}
<<Sprawozdanie z przeglądu Sprint'u -- czy założony cel (przyrost) został osiągnięty oraz czy wszystkie zaplanowane Backlog Item'y zostały zrealizowane? Demostracja przyrostu produktu>>.


\section{Sprint 4}

\subsection{Cel} Wdrożenie do aplikacji opcji "Szukaj" oraz systemu liczenia kalorii.

\subsection{Sprint Planning/Backlog}

\paragraph{Tytuł zadania.} Zaprogramowanie opcji "Szukaj".
\begin{itemize}
	\item Estymata: L.
\end{itemize}

\paragraph{Tytuł zadania.} Wdrożenie systemu liczenia kalorii.
\begin{itemize}
	\item Estymata: M.
\end{itemize}

\paragraph{<<Tutaj dodawać kolejne zadania>>}

\subsection{Realizacja}

\paragraph{Tytuł zadania.} Zaprogramowanie opcji "Szukaj".
\subparagraph{Wykonawca.} Poręba Bartłomiej, Zaczyk Bartłomiej, Szafrański Ireneusz.
\subparagraph{Realizacja.} <<Sprawozdanie z realizacji zadania (w tym ocena zgodności z estymatą). Kod programu (środowisko \texttt{verbatim}): \begin{verbatim}
for (i=1; i<10; i++)
...
\end{verbatim}>>.

\paragraph{Tytuł zadania.} Wdrożenie systemu liczenia kalorii.
\subparagraph{Wykonawca.} Poręba Bartłomiej, Zaczyk Bartłomiej, Szafrański Ireneusz.
\subparagraph{Realizacja.} <<Sprawozdanie z realizacji zadania (w tym ocena zgodności z estymatą). Kod programu (środowisko \texttt{verbatim}): \begin{verbatim}
for (i=1; i<10; i++)
...
\end{verbatim}>>.



\subsection{Sprint Review/Demo}
<<Sprawozdanie z przeglądu Sprint'u -- czy założony cel (przyrost) został osiągnięty oraz czy wszystkie zaplanowane Backlog Item'y zostały zrealizowane? Demostracja przyrostu produktu>>.


\section{Sprint 5}

\subsection{Cel} Wprowadzenie możliwości recenzji, oceny oraz zapisywania przepisów w pamięci urządzenia.

\subsection{Sprint Planning/Backlog}

\paragraph{Tytuł zadania.} Wprowadzenie systemu oceny i recenzji.
\begin{itemize}
	\item Estymata: M.
\end{itemize}

\paragraph{Tytuł zadania.} Wdrożenie opcji zapisywania przepisów offline.
\begin{itemize}
	\item Estymata: M.
\end{itemize}


\subsection{Realizacja}

\paragraph{Tytuł zadania.} Wprowadzenie systemu oceny i recenzji.
\subparagraph{Wykonawca.} Poręba Bartłomiej, Zaczyk Bartłomiej, Szafrański Ireneusz.
\subparagraph{Realizacja.} <<Sprawozdanie z realizacji zadania (w tym ocena zgodności z estymatą). Kod programu (środowisko \texttt{verbatim}): \begin{verbatim}
for (i=1; i<10; i++)
...
\end{verbatim}>>.

\paragraph{Tytuł zadania.} Wdrożenie opcji zapisywania przepisów offline.
\subparagraph{Wykonawca.} Poręba Bartłomiej, Zaczyk Bartłomiej, Szafrański Ireneusz.
\subparagraph{Realizacja.} <<Sprawozdanie z realizacji zadania (w tym ocena zgodności z estymatą). Kod programu (środowisko \texttt{verbatim}): \begin{verbatim}
for (i=1; i<10; i++)
...
\end{verbatim}>>.


\subsection{Sprint Review/Demo}
<<Sprawozdanie z przeglądu Sprint'u -- czy założony cel (przyrost) został osiągnięty oraz czy wszystkie zaplanowane Backlog Item'y zostały zrealizowane? Demostracja przyrostu produktu>>.



\begin{thebibliography}{9}

\bibitem{Cov} S. R. Covey, {\em 7 nawyków skutecznego działania}, Rebis, Poznań, 2007.

\bibitem{Oet} Tobias Oetiker i wsp., Nie za krótkie wprowadzenie do systemu \LaTeX  \ $2_\varepsilon$, \url{ftp://ftp.gust.org.pl/TeX/info/lshort/polish/lshort2e.pdf}

\bibitem{SchSut} K. Schwaber, J. Sutherland, {\em Scrum Guide}, \url{http://www.scrumguides.org/}, 2016.

\bibitem{apr} \url{https://agilepainrelief.com/notesfromatooluser/tag/scrum-by-example}

\bibitem{us} \url{https://www.tutorialspoint.com/scrum/scrum_user_stories.htm}

\end{thebibliography}

\end{document}

% ----------------------------------------------------------------
