\documentclass[a4paper]{article}
\usepackage{polski}
\usepackage[cp1250]{inputenc}
\usepackage{url}

\title{\bf{Kulinarna Aplikacja Mobilna}}
\author{{\em Poręba Bartłomiej, Zaczyk Bartłomiej, Szafrański Ireneusz}}
\date{}

\begin{document}

\begin{titlepage}
\maketitle
\thispagestyle{empty}
\bigskip
\begin{center}
Zespołowe przedsięwzięcie inżynierskie\\[2mm]

Informatyka\\[2mm]

Rok. akad. 2017/2018, sem. I\\[2mm]

Prowadzący: dr hab. Marcin Mazur
\end{center}
\end{titlepage}

\tableofcontents
\thispagestyle{empty}

\newpage

\section{Opis projektu}

\subsection{Członkowie zespołu}

\begin{enumerate}
\item Poręba Bartłomiej (kierownik projektu).
\item Zaczyk Bartłomiej.
\item Szafrański Ireneusz.
\end{enumerate}

\subsection{Cel projektu (produkt)}

Celem projektu jest stworzenie aplikacji mobilnej, która na podstawie posiadanych przez użytkownika produktów znajdzie danie które użytkownik będzie mógł sporządzić, a ponadto będzie miał wgląd do wartości kalorycznych potrawy, przeglądać zdjęcia gotowej potrawy, będzie mógł przeczytać recenzje na temat danej potrawy oraz posortować listę wyszukanych potraw wg. swoich upodobań.

\subsection{Potencjalny odbiorca produktu (klient)}

Klientem może być każdy kto nie ma pomysłu na obiad i/lub nie umie gotować. Aplikacja w prosty i przystępny dla każdego sposób krok po kroku opisze jak sporządzić wybraną potrawę.

\subsection{Metodyka}

Projekt będzie realizowany przy użyciu (zaadaptowanej do istniejących warunków) metodyki {\em Scrum}. 

\section{Wymagania użytkownika}
<<Przedstawić listę wymagań użytkownika w postaci ,,historyjek'' (User stories). Każda historyjka powinna opisywać jedną cechę systemu. Struktura: As a [type of user], I want [to perform some task] so that I can [achieve some goal/benefit/value] (zob. np. \cite{us}).>>

\subsection{User story 1}
Jako użytkownik aplikacji chcę mieć dostęp do prostego i czytelnego interfejsu, żebym mógł szybko i sprawnie znaleźć to czego szukam.
*Aplikacja dostosowana do różnych rozmiarów urządze? i systemów Android.
*Ograniczenie do minimum ilość przycisków.

\subsection{User story 2}
Jako użytkownik aplikacji chcę wiedzieć co mogę sporządzić z produktów, które posiadam, żebym nie tracił czasu na poszukiwania przepisów.
*Baza danych regularnie aktualizowana, aby zapewnić szeroki wybór przepisów.

\subsection{User story 3}
Jako użytkownik aplikacji chcę wiedzieć ile kalirii ma dana potrawa, żebym mógł kontrolować to co jem.
*Opcja liczenia kalorii dla pojedyńczego składniku i całej potrawy.

\subsection{User story 4}
Jako użytkownik aplikacji chcę sprawdzić jakich produktów brakuje mi do potrawy, żebym mógł delektować się jej smakiem.
*Opcja sortowania potraw wg ilości brakujących składników.

\subsection{User story 5}
Jako użytkownik aplikacji chcę wiedzieć jak inni użytkownicy oceniają daną potrawę, żebym mógł przygotować danie które oni polecają.
*Możliwość oceniania potraw i dzielenia się opinią z innymi użytkownikami aplikacji.

\subsection{User story 6}
Jako użytkownik aplikacji chcę dzielić się moimi przepisami, żebym mógł poznać opinie na jej temat.
*Możliwość dodawania przepisów do ogólnej bazy danych (po zatwierdzeniu).

\subsection{User story 7}
Jako użytkownik aplikacji chcę żeby aplikacja na telefon nie zajmowała dużo miejsca, żebym mógł mieć na nim miejsce na zdjęcia i filmy.
*Baza danych online, przez co mniejszy rozmiar aplikacji .

\subsection{User story 8}
Jako użytkownik aplikacji chcę wiedzieć ile czasu potrzebuje na sporządzenie danej potrawy, żebym mógł odpowiednio zagospodarować swój wolny czas.
*Przy każdej potrawie jest szacowany czas jej przygotowania.

\subsection{User story 9}
Jako użytkownik aplikacji chcę mieć dostęp do ulubionych przepisów kiedy jestem offline, żebym mógł korzystać z niej gdziekolwiek jestem
*Możliwość zapisywania ulubionych przepisów.

\subsection*{<<Tutaj dodawać kolejne historyjki>>}

\section{Harmonogram}

\subsection{Rejestr zadań (Product Backlog)}

\begin{itemize}
\item Data rozpoczęcia: 3.10.2017.
\item  Data zakończenia: 23.01.2018.
\end{itemize}

\subsection{Sprint 1}

\begin{itemize}
\item Data rozpoczęcia: <<data>>.
\item Data zakończenia: <<data>>.
\item Scrum Master: <<imię i nazwisko>>.
\item Product Owner: <<imię i nazwisko>>.
\item Development Team: <<lista developerów>>.
\end{itemize}

\subsection{Sprint 2}

\begin{itemize}
\item Data rozpoczęcia: <<data>>.
\item  Data zakończenia: <<data>>.
\item Scrum Master: <<imię i nazwisko>>.
\item Product Owner: <<imię i nazwisko>>.
\item Development Team: <<lista developerów>>.
\end{itemize}

\subsection*{<<Tutaj dodawać kolejne Sprint'y>>}

\section{Product Backlog}

\subsection{Backlog Item 1}
\paragraph{Tytuł zadania.} <<Tytuł>>.
\paragraph{Opis zadania.} <<Opis>>.
\paragraph{Priorytet.} <<Priorytet>>.
\paragraph{Definition of Done.} <<Określić (w języku zrozumiałym dla wszystkich członków zespołu), co oznacza ukończenie danego zadania>>.

\subsection{Backlog Item 2}
\paragraph{Tytuł zadania.} <<Tytuł>>.
\paragraph{Opis zadania.} <<Opis>>.
\paragraph{Priorytet.} <<Priorytet>>.
\paragraph{Definition of Done.} <<Określić (w języku zrozumiałym dla wszystkich członków zespołu), co oznacza ukończenie danego zadania>>.

\subsection*{<<Tutaj dodawać kolejne zadania>>}

\section{Sprint 1}
\subsection{Cel} <<Określić, w jakim celu tworzony jest przyrost produktu>>.
\subsection{Sprint Planning/Backlog}

\paragraph{Tytuł zadania.} <<Tytuł>>.
\begin{itemize}
\item Estymata: <<szacowana czasochłonność (w ,,koszulkach'')>>.
\end{itemize}

\paragraph{Tytuł zadania.} <<Tytuł>>.
\begin{itemize}
\item Estymata: <<szacowana czasochłonność (w ,,koszulkach'')>>.
\end{itemize}

\paragraph{<<Tutaj dodawać kolejne zadania>>}

\subsection{Realizacja}

\paragraph{Tytuł zadania.} <<Tytuł>>.
\subparagraph{Wykonawca.} <<Wykonawca>>.
\subparagraph{Realizacja.} <<Sprawozdanie z realizacji zadania (w tym ocena zgodności z estymatą). Kod programu (środowisko \texttt{verbatim}): \begin{verbatim}
for (i=1; i<10; i++)
...
\end{verbatim}>>.

\paragraph{Tytuł zadania.} <<Tytuł>>.
\subparagraph{Wykonawca.} <<Wykonawca>>.
\subparagraph{Realizacja.} <<Sprawozdanie z realizacji zadania (w tym ocena zgodności z estymatą). Kod programu (środowisko \texttt{verbatim}): \begin{verbatim}
for (i=1; i<10; i++)
...
\end{verbatim}>>.

\paragraph{<<Tutaj dodawać kolejne zadania>>}


\subsection{Sprint Review/Demo}
<<Sprawozdanie z przeglądu Sprint'u -- czy założony cel (przyrost) został osiągnięty oraz czy wszystkie zaplanowane Backlog Item'y zostały zrealizowane? Demostracja przyrostu produktu>>.

\section{Sprint 2}

\subsection{Cel} <<Określić, w jakim celu tworzony jest przyrost produktu>>.

\subsection{Sprint Planning/Backlog}

\paragraph{Tytuł zadania.} <<Tytuł>>.
\begin{itemize}
\item Estymata: <<szacowana czasochłonność (w ,,koszulkach'')>>.
\end{itemize}

\paragraph{Tytuł zadania.} <<Tytuł>>.
\begin{itemize}
\item Estymata: <<szacowana czasochłonność (w ,,koszulkach'')>>.
\end{itemize}

\paragraph{<<Tutaj dodawać kolejne zadania>>}

\subsection{Realizacja}

\paragraph{Tytuł zadania.} <<Tytuł>>.
\subparagraph{Wykonawca.} <<Wykonawca>>.
\subparagraph{Realizacja.} <<Sprawozdanie z realizacji zadania (w tym ocena zgodności z estymatą). Kod programu (środowisko \texttt{verbatim}): \begin{verbatim}
for (i=1; i<10; i++)
...
\end{verbatim}>>.

\paragraph{Tytuł zadania.} <<Tytuł>>.
\subparagraph{Wykonawca.} <<Wykonawca>>.
\subparagraph{Realizacja.} <<Sprawozdanie z realizacji zadania (w tym ocena zgodności z estymatą). Kod programu (środowisko \texttt{verbatim}): \begin{verbatim}
for (i=1; i<10; i++)
...
\end{verbatim}>>.

\paragraph{<<Tutaj dodawać kolejne zadania>>}


\subsection{Sprint Review/Demo}
<<Sprawozdanie z przeglądu Sprint'u -- czy założony cel (przyrost) został osiągnięty oraz czy wszystkie zaplanowane Backlog Item'y zostały zrealizowane? Demostracja przyrostu produktu>>.

\section*{<<Tutaj dodawać kolejne Sprint'y>>}


\begin{thebibliography}{9}

\bibitem{Cov} S. R. Covey, {\em 7 nawyków skutecznego działania}, Rebis, Poznań, 2007.

\bibitem{Oet} Tobias Oetiker i wsp., Nie za krótkie wprowadzenie do systemu \LaTeX  \ $2_\varepsilon$, \url{ftp://ftp.gust.org.pl/TeX/info/lshort/polish/lshort2e.pdf}

\bibitem{SchSut} K. Schwaber, J. Sutherland, {\em Scrum Guide}, \url{http://www.scrumguides.org/}, 2016.

\bibitem{apr} \url{https://agilepainrelief.com/notesfromatooluser/tag/scrum-by-example}

\bibitem{us} \url{https://www.tutorialspoint.com/scrum/scrum_user_stories.htm}

\end{thebibliography}

\end{document}

% ----------------------------------------------------------------
